\usepackage[PunctStyle=plain,RubberPunctSkip=false]{xeCJK}
\XeTeXlinebreaklocale "zh"
\XeTeXlinebreakskip = 0pt plus 1pt

\usepackage[OT1]{fontenc}
\usepackage[dvipsnames]{xcolor}
\usepackage[hidelinks]{hyperref}
\hypersetup{colorlinks=false}
\usepackage{calc,fontspec,tcolorbox,paralist,enumitem,tocloft}



\setlength{\parindent}{0pt}



% ==========================================
% Functions
% ==========================================
\newcommand{\uipage}[2]{
    % $1=caption $2=content
    % \begin{minipage}[t][\textheight-4em][t]{\textwidth}
    %     #2
    % \end{minipage}
    \begin{tcolorbox}[height=\textheight-3em,width=\textwidth,arc=0mm,top=0mm,bottom=0mm,left=0mm,right=0mm,nobeforeafter,colframe=black,colback=white,boxrule=0.5pt]
        #2
    \end{tcolorbox}\par
    \vfill
    \begin{minipage}{\linewidth}
        \center\sffamily\normalsize\mdseries#1
        \addcontentsline{toc}{section}{#1}
    \end{minipage}

    \clearpage
}











\makeatletter










% ==========================================
% Dimensions
% ==========================================
\newlength{\uicdim@topbarheight}
\newlength{\uicdim@buttonwidth}
\newlength{\uicdim@buttonheight}









% ==========================================
% Predefined Component Slot Content
% ==========================================
\newcommand{\UICSlotContentTopbar}[0]{
    This is a topbar
}





% ==========================================
% Call Components
% ==========================================
\newcommand{\uicTopbar}[0]{
    % $1=content
    \settoheight{\uicdim@topbarheight}{\UICSlotContentTopbar}
    \begin{minipage}{\linewidth}
        \vspace{1mm}
        \UICSlotContentTopbar
        \vspace{2mm}
    \end{minipage}\par\hrule
    % \begin{tcolorbox}[height=\uicdim@topbarheight+2mm,width=\linewidth,arc=0mm,top=0mm,bottom=0mm,left=0mm,right=0mm,nobeforeafter,colback=white,colframe=red,boxrule=0.5pt]
    %     \UICSlotContentTopbar\par
    %     \vspace{2mm}
    %     \par\hrule
    % \end{tcolorbox}
}


% ==========================================
% Small Components
% ==========================================
\newcommand{\uiButtonCore}[4]{
    % $1=content $2=colback $3=colframe $4=colortext
    % \settowidth{\uicdim@buttonwidth}{\footnotesize#1}
    % \settoheight{\uicdim@buttonheight}{\footnotesize#1}
    % \begin{tcolorbox}[height=\uicdim@buttonheight+3mm,width=\uicdim@buttonwidth+5mm,arc=1mm,top=0mm,bottom=0mm,left=0mm,right=0mm,nobeforeafter,colback=#2,colframe=#3,boxrule=0.5pt]
    %     \center\footnotesize{\textcolor{#4}{#1}}
    % \end{tcolorbox}
    \fbox{{\small#1}}
}
\newcommand{\uiButton}[1]{
    \uiButtonCore{#1}{white}{black}{black}
}




\makeatother
