\usepackage[PunctStyle=plain,RubberPunctSkip=false]{xeCJK}
\XeTeXlinebreaklocale "zh"
\XeTeXlinebreakskip = 0pt plus 1pt

\usepackage[OT1]{fontenc}
\usepackage[dvipsnames]{xcolor}
\usepackage[hidelinks]{hyperref}
\hypersetup{colorlinks=false}
\usepackage{calc,fontspec,tcolorbox,paralist,enumitem,tocloft}



\setlength{\parindent}{0pt}


















\makeatletter











% ==========================================
% Dimensions
% ==========================================
\newlength{\uicdim@topbarheight}
\newlength{\uicdim@buttonwidth}
\newlength{\uicdim@buttonheight}


% Canvas dimensions
\newlength{\uiwire@RootCanvasWidth}
\newlength{\uiwire@RootCanvasHeight}
\newlength{\uiwire@canvasWidth}
\newlength{\uiwire@canvasHeight}

% Swap variables
\newlength{\uiwire@tmpCanvasWidth}
\newlength{\uiwire@tmpCanvasHeight}












% ==========================================
% Functions
% ==========================================
\newcommand{\uipage}[2]{
	% $1=caption $2=content
    \setlength{\uiwire@RootCanvasWidth}{\textwidth}
    \setlength{\uiwire@RootCanvasHeight}{\textheight-3em}
    \clearpage
	\begin{tcolorbox}[height=\uiwire@RootCanvasHeight+6pt,width=\uiwire@RootCanvasWidth,arc=0mm,top=0mm,bottom=0mm,left=0mm,right=0mm,nobeforeafter,colframe=black,colback=white,boxrule=0.8pt]
        \setlength{\uiwire@canvasWidth}{\uiwire@RootCanvasWidth}
        \setlength{\uiwire@canvasHeight}{\uiwire@RootCanvasHeight}
		#2
	\end{tcolorbox}\par
	\vfill
	\begin{minipage}{\linewidth}
		\center\sffamily\normalsize\mdseries#1
		\addcontentsline{toc}{section}{#1}
	\end{minipage}
	\clearpage
}

\newcommand{\leftNright}[3]{%
    % $1=leftWidth $2=leftContent $3=rightContent
    \setlength{\uiwire@tmpCanvasWidth}{\uiwire@canvasWidth}%
    \setlength{\uiwire@tmpCanvasHeight}{\uiwire@canvasHeight}%
    % Step 1: Create Left Box
    \begin{minipage}[t][\uiwire@tmpCanvasHeight][t]{#1}%
        \setlength{\uiwire@canvasWidth}{#1}%
        \setlength{\uiwire@canvasHeight}{\uiwire@tmpCanvasHeight}%
        #2%
    \end{minipage}\hfill\vrule\hfill%
    % Step 2: Create Right Box
    \begin{minipage}[t][\uiwire@tmpCanvasHeight][t]{\uiwire@canvasWidth-#1-12pt}%
        \setlength{\uiwire@canvasWidth}{\uiwire@canvasWidth-#1-12pt}%
        \setlength{\uiwire@canvasHeight}{\uiwire@tmpCanvasHeight}%
        #3%
    \end{minipage}%
}
\newcommand{\topNbottom}[3]{%
    % $1=topHeight $2=topContent $3=bottomContent
    \setlength{\uiwire@tmpCanvasWidth}{\uiwire@canvasWidth}%
    \setlength{\uiwire@tmpCanvasHeight}{\uiwire@canvasHeight}%
    % Step 1: Create Top Box
    \begin{minipage}[t][#1][t]{\uiwire@canvasWidth}%
        \setlength{\uiwire@canvasWidth}{\uiwire@tmpCanvasWidth}%
        \setlength{\uiwire@canvasHeight}{#1}%
        #2%
    \end{minipage}\vfill\hrule\vfill%
    % Step 2: Create Bottom Box
    \begin{minipage}[t][\uiwire@tmpCanvasHeight-#1-12pt][t]{\uiwire@canvasWidth}%
        \setlength{\uiwire@canvasWidth}{\uiwire@tmpCanvasWidth}%
        \setlength{\uiwire@canvasHeight}{\uiwire@canvasHeight-#1-12pt}%
        #3%
    \end{minipage}%
}














% ==========================================
% Canvas Management
% ==========================================

\newcommand{\DocumentationForCanvas}[0]{
    \parskip=8pt
    \textbf{\Large{}Documentation For Canvas}

	We use a ``Canvas'' system in our wireframe graphs.
    A ``uipage'' creates a ``root canvas'', whose size is automatically determined.
    
    As we use ``leftNright'', the current canvas is split into two canvases horizontally.

    Similarly, when we use ``topNbottom'', the current canvas is split into two canvases vertically.

    By invoking any of the two commands, we pass three arguments:

    \begin{compactitem}
        \item Top subcanvas height or left subcanvas width
        \item Content of the first subcanvas
        \item Content of the second subcanvas
    \end{compactitem}

}



















% ==========================================
% Predefined Component Slot Content
% ==========================================
\newcommand{\UICSlotContentTopbar}[0]{
	This is a topbar
}





% ==========================================
% Call Components
% ==========================================
\newcommand{\uicTopbar}[0]{
	% $1=content
	\settoheight{\uicdim@topbarheight}{\UICSlotContentTopbar}
	\begin{minipage}{\linewidth}
		\vspace{1mm}
		\UICSlotContentTopbar
		\vspace{2mm}
	\end{minipage}\par\hrule
	% \begin{tcolorbox}[height=\uicdim@topbarheight+2mm,width=\linewidth,arc=0mm,top=0mm,bottom=0mm,left=0mm,right=0mm,nobeforeafter,colback=white,colframe=red,boxrule=0.5pt]
	%     \UICSlotContentTopbar\par
	%     \vspace{2mm}
	%     \par\hrule
	% \end{tcolorbox}
}


% ==========================================
% Small Components
% ==========================================
\newcommand{\uiButtonCore}[4]{
	% $1=content $2=colback $3=colframe $4=colortext
	% \settowidth{\uicdim@buttonwidth}{\footnotesize#1}
	% \settoheight{\uicdim@buttonheight}{\footnotesize#1}
	% \begin{tcolorbox}[height=\uicdim@buttonheight+3mm,width=\uicdim@buttonwidth+5mm,arc=1mm,top=0mm,bottom=0mm,left=0mm,right=0mm,nobeforeafter,colback=#2,colframe=#3,boxrule=0.5pt]
	%     \center\footnotesize{\textcolor{#4}{#1}}
	% \end{tcolorbox}
	\fbox{{\small#1}}
}
\newcommand{\uiButton}[1]{
	\uiButtonCore{#1}{white}{black}{black}
}




\makeatother
